\chapter*{Introduction and~motivation}
\addcontentsline{toc}{chapter}{Introduction and~motivation}
\label{ch:preface}
Nowadays, there are many techniques of~processing data. Unfortunately, there are many
different data formats one can work with. It~makes the~processing a~lot more difficult.
The task becomes even harder when one wants to~connect two different datasets in~order to
benefit from the~connection. The~connection allows us~to~get some additional information
about entities from each of~the~standalone datasets. Therefore, a~lot of~computation time is~spent
on converting, formatting and~transforming data into~another form. But transforming datasets
into a~matching format is~not enough. One needs to~specify how the~data should be~linked
together.

There are many ways of~doing that. Starting with implementing the~logic into~a~simple conversion
script (according to~a~specific dataset) to~introducing a~more complex metadata description
framework for purposes of~generic data processing. Since one of~the~most attractive
tasks in~this area is~to~be~able to~connect any~of~the~datasets available on~the~Internet,
we are interested in~the~generic description frameworks. We~would like to~have 
a~tool, 
which enables us~to~work with any~data on~the~Internet (formatted according to~some
kind of~rules). We~would like to~link them together, analyze them and~visualize 
them.

One of~the~most used description frameworks is~the~Resource Description Framework~\cite{rdf}.
It is~a~standard model for data interchange on~the~Web. It~tells us~how to~describe
resources on~the~Internet in~order to~allow other people, applications and~tools
to understand such a~description. That gives us~a~potential to~link any~data on~the~Internet.
Based on~the~framework, a~new model named Linked Data~\cite{ld} was introduced. The~model
has been brought up~to~make data interconnecting easier.

The result of~interconnecting data while utilizing the~principles of~the~Linked Data model
and Resource Description Framework is~a~directed graph. Its vertices represent resources we
have information about. The~edges stand for relations between such entities. From this point on,
it is~up~to~us, how we~look at~the~data. We~can either explore them in~a~plain graph or~apply
some more semantics and~make domain specific visualizations while using ontologies
and other advanced techniques.

One of~the~specific domains are statistical data, which are one of~the~most interesting kind
of data. They are produced and~processed by~many stakeholders. In~the~context of
\emph{Linked Open Data}, the~most interesting are, of~course, governments and~scientific groups.
But we~would like to~work with such data in~the~usual
way --- make tables, charts or~more interesting visualizations. While speaking about Open Data, a~specific
user group --- data journalists --- would like to~work with Linked Data, but they are probably
missing some basic tools, which would enable them to~interpret gathered results
in the~way readers would understand.

After applying the~rules of~the~Linked Data model, the~statistical data (even tabular data)
get transformed into~a~generic graph. The~only, but very important advantage, is~that we~have some additional
metadata information available. Moreover, the~data are still linked with related entities from all over the~Internet.
That brings
us to~another model, the~Data Cube Vocabulary~\cite{dcv}. It~is~a~model, which tells us~how to~describe
multi-dimensional (statistical) data with~respect to~Linked Data and~RDF 
principles.

\section*{Goals of~the~thesis}

The aim of~this thesis is~to~describe these models, analyze the~possibilities of~working
with multi--dimensional data in~the~environment of~Linked Data. We~will also~propose a~system
which will enable its user to~convert Linked Data into~the~Data Cube Vocabulary model.
A prototype of~such a~system will be~implemented. An~exemplary visualisation of~the~statistical
data will be~implemented and~presented. Moreover, some missing Payola user interface features 
will be~implemented.

\section*{Structure of~the~text}
In Chapter~\ref{ch:statistical-data}, we~describe the~aforementioned 
models and~standards --- RDF, Linked Data, Data Cube and~Data Cube Vocabulary. 
Some examples are presented. Chapter~\ref{chap:rw} contains description of~existing tools and~applications. We~compare those to~our application, Payola~\cite{payola}.
We also examine some related papers, especially the~LDVM proposal~\cite{ldvm}. 
The Payola application is~described in~Chapter~\ref{ch:payola}. Later on~(Chapter~\ref{ch:proposal}),
we propose a~system
for analyzing and~visualizing data compliant with the~Data Cube Vocabulary model.
Implementation of~a~prototype is~described in~Chapter~\ref{ch:implementation}.
In Chapter~\ref{ch:experiments}, we~present capabilities of~the~implemented system
and~experiment with some statistical datasets.