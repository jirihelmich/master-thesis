\chapter{System proposal}
\label{ch:proposal}
Based on~what we~have presented in~Chapter~\ref{chap:rw}, we~would like to
propose a~system, which will enable the~user to~map arbitrary RDF dataset
into a~form compliant with the~Data Cube Vocabulary standard. Since this is~quite an~open definition, we~have to~dive into~the~problem a~little bit deeper to~find out what is~possible and to~propose a~system that meets some real--world 
functional requirements.

We are about to~propose a~system that is~in a~specific way very similar to~the implementation of~OLAP2DataCube (Section~\ref{rw:olap2dc}) or~Tabels (Section~\ref{rw:tabels}).
On the~other hand, both of~them are 
focused on~converting arbitrary statistical \emph{non--RDF} data into~a~form 
compliant with the~Data Cube Vocabulary. In~Figure~\ref{fig:olap2dc-mapping} we~remind
the reader how the~transformation process is~done in~the case of~OLAP2DataCube.
The important fact is~that the~data source (RDBMS) already contains statistical data in~an obvious form. 
Those are structured in~tables connected via foreign keys and organized in~a 
shape of~a star or~a snowflake. The~system is~designed to~take advantage of~those 
foreign keys. Based on~them it~categorizes tables (fact table, dimension tables) and hints
the user within the~cube definition step.
The user is~required to~define relations semantics and select columns 
containing measures and dimensions. In~the last step, a~mapping is~done based on~the information gathered in~the previous steps.
Notice that a~Data Cube Vocabulary definition is~a part of~the result.


\begin{figure}
	\centering
	\includegraphics[width=140mm]{img/mapping-olap2dc.png}
	\caption{OLAP2DataCube mapping process}
	\label{fig:olap2dc-mapping}
\end{figure}


Let us~take a~brief excursion back to~the OLAP2DataCube described in~Section 
~\ref{rw:olap2dc}. The~tool was not designed just to~convert a~non--RDF dataset 
into the~RDF standard with respect to~the DCV metaformat. Executing the~tool on~a 
dataset has one important side--effect: there is~a new QB~data structure definition
generated with bases on~the structure of~the input dataset. Therefore, the~tool does not map the~dataset to~match an~existing data structure definition, it~creates a~new one. That is~definitely
needed when one transforms a~non--RDF dataset into~RDF. But once 
this is~done, it~is probable that a~company or~an organzation will produce the~same kind of~data periodically, therefore, they will reuse the~same data structure definition.

This is~the reason why we~will propose a~system which fill focus on~the task of~mapping the~input dataset to~match an~existing data structure definition. 
The authors of~the LDVM~\cite{ldvm} (see Section~\ref{sec:rw:ldvm}) propose to~make standalone
visualizers. Each of~them will 
be able to~visualize a~specific kind of~datasets. Its visualization abilities will be~described with
an \emph{input signature}. A~Data Cube data structure definition could be~easily transformed into
an example of~such a~signature. That is~another reason why we~want the~tool to~map the~input dataset into~an~existing definition.

\begin{figure}
	\centering
	\includegraphics[width=140mm]{img/generic-mapping.png}
	\caption{RDF to~DataCube mapping in~a generic system}
	\label{fig:generic-mapping}
\end{figure}

That means we~would like to~enable the~user to~take an~existing RDF
dataset and give it~a statistical meaning. The~initial thought was to
implement a~set of~specialized 
mapping modules. We~would have to~implement a~standalone mapping module for
each existing Data Cube Vocabulary. Since this is~rather impossible and 
surely inefficient, we~made some experiments in~order to~ensure that we~would be~able to~come 
up with a~generic mapping component. That is~why the~process of~the implemented
system will be~similar to~what is~shown in~Figure~\ref{fig:generic-mapping}. The~process seen in~the diagram represents a~high--level view of~the system. As~we 
consider some additional criteria we~will provide a~more detailed schema of~the 
proposed system. 

The biggest difference is~that the~original dataset can contain the~statistical 
data in~a not--so--obvious form, as~in the~case of~the relational DB.
The user may need to~apply an~analytical 
extraction (a term introduced in~\cite{ldvm}) in~order to~select the~transformed data.
We start with an~arbitrary RDF document, 
specify the~process of~mapping and how the~system transforms the~data in~order
to comply with the~Data Cube Vocabulary standard. We~also require the~user to~provide a~Data Cube Vocabulary definition. The~proposed system will map the~data
to conform with such a~definition.

We will now walk the~reader through the~decision process we~have made in~order to~propose 
the system. As~stated before, the~input is~an arbitrary RDF graph. To~be able to~map data from the~graph to~a form compliant with a~Data 
Cube Vocabulary, the~system also needs for a~vocabulary to~be a~part of~the input. 
As discovered in~Section~\ref{datacube-vocabulary}, the~vocabulary is~another 
RDF graph.

Based on~that fact, we~concur that the~system needs to~contain a~mechanism, which 
parses an~arbitrary RDF graph and searches for DCV data structure definitions. 
We need to~extract those definitions and have the~user select one. The~selected 
definition will be~used in~the upcoming steps to~obtain mapping specification from 
the user. What we~need to~obtain is~presented in~Figure~\ref{fig:mapping-example}. On~the right side the~reader can see the~generic visualization of~a DCV data structure definition. A~random graph is~placed on~the left side of~the illustration. The~dashed lines connect the~resources 
from the~original dataset with corresponding DCV components. Naturally, the~figure represents an~example of~a mapping.

\begin{figure}
	\centering
	\includegraphics[width=140mm]{img/mapping-example.png}
	\caption{Mapping example. Original dataset on~the left, DCV data structure definition on~the right.}
	\label{fig:mapping-example}
\end{figure}

\section{Pattern selection}
We are lacking some rules telling the~system how to~do the~mapping. 
Without such rules it~is not possible to~map the~input graph properly. Let us~imagine an~input graph, for instance the~whole DBPedia database. That represents a~huge graph 
containing many facts and naturally, some of~those are statistical data. But they 
are not referenced in~such a~context. For example the~resource \emph{Prague}, which is~the 
capital city of~the Czech Republic has properties \texttt{populationTotal} and
\texttt{populationAsOf}. Together it~creates a~slice of~a cube. This says that the~city Prague (location dimension) had in~a certain day (time dimension) 
a certain amount of~citizens (measure).

Although this cube is~originally based on~a very simple pattern (a shape of~a pair of~cherries),
it is~not possible to~map such a~slice into~a~form compliant with a~data structure definition since the~relation between the~definition and the~dataset is~not defined in~the original data. For a~well--defined data structure 
definition and a~well--defined dataset, it~could be~possible to~propose another 
system, which would try to~detect the~relation and create them 
automatically. For instance, the~location dimension, which is~usually named
\emph{reference area} could be~connected to~a generic type -- a~location.
Also the~resource \emph{Prague} could be~connected with the~location type. But 
that would mean introducing a~system, which would process a~lot of~data trying to~match each supertype of~the dimension with the~properties of~every resource in~the graph. It~would have to~use advanced heuristics and it~would probably have 
to adopt some techniques from the~machine--learning field. A~fully--automated 
system would also introduce a~lot of~mistakes into~the~results since it~would 
be a~highly experimental approach. Therefore, implementing such a~system would 
not be~the goal of~this thesis. 

On the~other hand, we~would like to~propose a~system, which will be~flexible 
enough to~allow the~user to~take an~arbitrary Data Cube Vocabulary data 
structure definition and map a~reasonable part of~this dataset into~a~compliant 
form.

Unfortunately, the~phrase \emph{reasonable part} is~not very scientific. 
It comes from the~data semantics and it~is quite difficult to~formalize. What 
the user should map to~a given data structure definition should have 
a corresponding semantics. For instance, if~we use the~population size example, the~user should probably not map the~statistics of~government debts to~that 
definition. Even though a~country is~a location, the~amount of~the debt represents a~number and was certainly calculated on~a specific day, therefore, the~data match the~ranges of~the data structure definition components, but it~does 
not match the~semantics.

\begin{sloppypar}
On the~other hand, we~do not want to~make the~tool too restrictive in~order to~achieve better user experience. For instance, we~have mentioned
(Figure~\ref{fig:example-dcv-dataset}) a~data structure 
for the~population example that declares a~dimension \texttt{czso-ds-def:refArea}. 
The range of~that dimension is~specified as~\texttt{czso-reg:Municipality}. 
With a~restrictive approach, the~tool would not enable the~user to~map the~data from 
the DBPedia database to~match the~data structure definition, since the~DBPedia 
dataset does not contain the~\texttt{czso-reg} namespace. Nonetheless, Prague is~certainly a~Czech municipality. Therefore, the~tool will not check the~range 
while performing the~mapping. If~it had, the~user would need to~involve 
other datasets so~as to~provide the~range--related information.
\end{sloppypar}

What is~still missing is~the way of~informing the~tool about how to~map the~data.
Obtaining such a~specification can be~implemented in~many different ways. 
All of~them are actually described in~Chapter~\ref{chap:rw}. Some of~the tools used 
their native language -- the~SPARQL. That is~beneficial for a~developer due to~its relatively easy implementation. But it~is a~bit cumbersome for a~user 
who is~required to~know the~SPARQL language. On~the other hand, from all the~approaches
it restricts the~possibilities of~the system the~least. It~enables the~user 
to specify a~wide range of~mappings.

OLAP2DataCube took an~advantage of~the database structure (Section ~\ref{olap2dc}) --- 
especially of~relational feature --- foreign keys. Since the~RDF format 
is based on~expressing relations it~would not present an~issue. But what the~OLAP2DataCube tool did, was that it~transformed already existing statistical data from
the star--shaped database structure into~another format. But in~the world of~RDF 
data we~would restrict the~tool a~lot if~we relied only on~star--shaped 
schemas, hence, we~need to~work with a~generic graph.

Tabels introduced a~custom DSL that should have made the~situation easier for 
a user (Section~\ref{rw:tabels}). But as~experienced while doing research for Chapter~\ref{chap:rw}, 
we did not find it~very user--friendly. There certainly was a~pre--generated script based
on the~uploaded file but in~most cases the~user had to~modify the~script in~order to
achieve desired results. The~user is~required to~learn a~completely new language, but will probably not find any other use of~such a~skill
elsewhere. Moreover, it~is not an~easy task to~implement the~custom DSL itself. Therefore, we~consider this approach to~be the~worst.

The last approach we~have noted is~the \emph{query--by--example} introduced in~Tabulator (Section~\ref{sec:rw:tabulator}). It~was used to~let the~user specify what type of~data patterns they are 
interested in. In~the exploration mode the~user was able to~\emph{visually} 
specify the~pattern and switch to~the analytical mode. Such an~approach enables 
us to~construct a~SPARQL query and come closer to~the very first (and the~least restrictive) 
approach. Nonetheless, it~brings some new implementation issues.

The most obvious one is~the transformation of~the user's selection into~a~SPARQL query. 
We will also require the~system to~be capable of~making a~preview of~the 
original dataset in~order to~offer the~user a~visualization. The~visualization 
itself will interact with the~user and provide them a~way to~specify the~pattern.

Of course, the~query--by--example does not need to~be necessarily done visually. But 
obtaining an~example with a~non--visual approach leads to~the involvement of~a DSL. Even 
a simple DSL, for instance a~variation of~CSV format, is~not as~user--friendly as~the visual approach. The~visual approach also enables us~to force some 
restrictions to~avoid dealing with an~arbitrary user input. We~will 
obtain an~input that is~already valid.

After obtaining the~pattern all the~necessary information are gathered. We~may proceed to~the mapping process itself. The~mapping module will query the~data source for the~data. It~will use the~pattern specified by~the user. After 
fetching the~data, it~will construct a~new graph containing a~set of~DCV 
observations. Based on~what we~have learnt so~far, the~system will appear in~detail 
as shown in~Figure~\ref{fig:generic-mapping-detail}.

\begin{figure}
	\centering
	\includegraphics[width=140mm]{img/generic-mapping-detail.png}
	\caption{Detail of~the proposed system.}
	\label{fig:generic-mapping-detail}
\end{figure}


\section{Mockups}
\FloatBarrier
We will show the~reader our idea of~a user interface of~the proposed system. We~will refine some previously outlined features, especially the~pattern selection.

In the~first step (Figure~\ref{fig:mockup-01}), the~user is~required to~specify the~source of~the data that 
will be~transformed. They will probably supply a~reference to~a SPARQL Endpoint,
e.g. \url{http://dbpedia.org/sparql}. The~user also has to~reference a~graph, which contains the~desired Data Cube Vocabulary data structure definition.
That is~a URL as~well, e.g. \url{http://datacube.payola.cz/dsd/population.ttl}.
The type of~data source 
is needed in~order to~determine how to~fetch the~data.
\begin{figure}
	\centering
	\includegraphics[width=120mm]{img/mockup-01.png}
	\caption{Step 1: data source and vocabulary specification}
	\label{fig:mockup-01}
\end{figure}

After proceeding to~the next step (Figure~\ref{fig:mockup-02}),
the user will be~asked to~choose a~DCV data 
structure definition. The~definition list is~prepared based on~the vocabulary provided 
in the~previous step. The~system has already parsed the~supplied graph and filtered all
DCV data structure definitions. As~an example, the~user will select the
\url{http://datacube.payola.cz/dsd#PopulationSizeDefinition} data structure definition.

\begin{figure}
	\centering
	\includegraphics[width=120mm]{img/mockup-02.png}
	\caption{Step 2: DCV data structure selection}
	\label{fig:mockup-02}
\end{figure}

The system now selects data for a~preview. The~retrieved dataset is~visualized and shown in~a dialog (Figure~\ref{fig:mockup-03}). The~user is~required to~specify a~pattern. The~specification is~done by~clicking the~vertices in~the visualized graph (some progress is~shown in~Figure~\ref{fig:mockup-05}).

\begin{figure}
	\centering
	\includegraphics[width=120mm]{img/mockup-03.png}
	\caption{Step 3: Pattern selection}
	\label{fig:mockup-03}
\end{figure}
\begin{figure}
	\centering
	\includegraphics[width=120mm]{img/mockup-05.png}
	\caption{Step 3 (example): Pattern selection. The~user has to~provide an~example
	for mapping the~dataset to~a population vocabulary. In~the last step, the~user is~asked
	to select a~vertex corresponding to~the \texttt{payola-dcv:populationSize} property.
	They have already selected the~vertices representing Prague and observation date
	in order to~satisfy the~other components. The~\texttt{dbpedia-owl:City} vertex can
	be selected to~refine the~example, but it~does not correspond with any component.
	The user is~about to~select the~vertex connected via the~\texttt{populationAsOf} 
	relation. This represents an~example of~selecting a~meaningful pattern, which 
	is a~connected graph.}
	\label{fig:mockup-05}
\end{figure}

After the~pattern is~specified the~system has all the~information it~needs. The~user is~able to~start the~mapping process. When done, the~system offers
the results (RDF graph).
\FloatBarrier

\section{Benefits of~integration into~Payola}
\label{why-payola}
Since we~have participated on~the implementation of~the Payola RDF tool, we~will 
now go~through the~pros and cons of~integrating the~proposed system into~Payola.
We will also decide if~we proceed with the~integration.

In Chapter~\ref{ch:payola} we~familiarized the~reader briefly with the~main
concepts of~the~Payola framework. The~reader also knows how the~analyses
evaluation is~done. Therefore we~may demonstrate the~benefits arising from
the integration of~the proposed system with the~Payola framework.

The crucial feature of~the Payola framework is~the analysis subsystem. It~enables a~user
to combine multiple data sources and benefit from their 
combination. Based on~that, a~completely new set of~facts can be~computed.
(This could be~optimized with SPARQL 1.1 and Federated queries~\cite{federated-queries}.) 
Therefore, it~could be~handy to~introduce a~Data Cube Vocabulary analytical 
plugin, which would be~able to~convert the~results of~an analysis into~the~Data 
Cube Vocabulary format. Such a~plugin could be~connected directly to~an output
of a~data fetcher plugin --- this will cover the~basic implementation shown in~Figure~\ref{fig:generic-mapping-detail}.

But we~can go~a bit further. Since it~would be~an analytical plugin, the~user 
would be~able to~use such a~plugin in~an arbitrary step of~the analysis. Let us~look back again
at the~DBPedia and the~population statistics example. A~plugin representing the~proposed
system would substitute multiple different plugins 
in order to~obtain the~same data. In~this case it~is the~\texttt{Typed}
plugin and the~\texttt{Property selection} plugin. Those are covered based on~the selected pattern.
In addition, the~result would be~compliant with the~Data Cube Vocabulary standard.

Another asset is~considered to~be the~ability to~transform multiple data 
sources in~one step. The~users might also further use the~converted datasets --- combine them
(join, union) and analyze those datasets in~a unified format (specified by~the DCV).

The next benefit of~the integration with the~Payola analyzer is~that the~user 
can assemble an~algorithm, which is~applicable multiple times. The~main 
advantage is~that it~reflects the~current state of~the data sources at
the time of~the execution. That means that if~the data source contents get 
updated, the~user just executes the~analysis again and the~mapping is~instantly 
performed. There is~no need to~specify anything for the~second time.
This is~useful especially in~a case where the~analysis is~more complicated than the~mapping process itself.

As stated before, the~Payola application also offers the~functionality of~sharing. This feature is~usable at~least in~a case of~sharing the~results 
of an~analysis. The~introduction of~a Data Cube Vocabulary plugin does not change anything.
The user will still be~able to~share the~analysis. If~the analysis contains just the~data fetcher 
and the~DCV plugins, the~user shares only the~mapping process.

Another aspect could be~an effort needed to~implement such a~system. Since the~goal of~software engineering is~not to~implement the~same systems over and over 
but to~bring new systems, new features, speed up~the computation process and 
bring better user experience, it~makes no~sense not to~take advantage of~an already existing platform. The~same approach could be~seen in~the case of~reviewed 
tools like Olap2DataCube (section~\ref{olap2dc}) or~CubeViz 
(section~\ref{cubeviz}).

The Payola framework contains a~variety of~useful components. As~an example of~such components we~might name data fetchers, RDF processing modules or~visualization API.

Since a~part of~the goal of~this thesis is~to implement an~exemplary 
visualization that takes advantage of~the Data Cube Vocabulary format, it~will 
be very useful to~stay focused on~the task and implement just the~visualizer, 
instead of~a lot of~supporting subsystems that are already 
implemented elsewhere.

Moreover, we~can take advantage of~the existing visualization plugins. The~basic 
one, triple table plugin, offers a~quick overview of~the data. The~user 
will also be~able to~download the~results of~the transformation into~a~static 
RDF file for further use or~just for simple backup purposes.

\begin{figure}
	\centering
	\includegraphics[width=140mm]{img/payola-mapping.png}
	\caption{Mapping process integrated into~the~Payola analyzer.}
	\label{fig:payola-mapping}
\end{figure}

We show a~possible way of~the integration of~the mapping process into~the~Payola analytical extraction
in Figure~\ref{fig:payola-mapping}. It~shows the~proposed system in~a context of~an
analytical pipeline. Such a~pipeline is~constructed by~the user. The~process starts with querying data 
sources in~order to~offer the~user an~overview. The~amount of~the sources depends 
on which sources are involved in~the analytical pipeline. They are fully independent 
and spread throughout the~whole Internet. The~obtained data are visualized to~the user who is~required to~select a~pattern. Since we~need to~guide the~user 
to select a~proper one, we~also need them to~specify a~Data Cube Vocabulary. 
We will need to~parse the~Vocabulary, extract the~DCV datastructure definitions 
and offer the~user a~list of~the available definitions. After they choose one, we~will be~able to~determine how many mapping references we~need the~pattern to~contain. The~number is, of~course, based on~the amount of~components 
within the~datastructure definition. We~will construct a~SPARQL query based on~the selected pattern.

The rest of~the process is~a common analytical pipeline execution. The~Data Cube 
Vocabulary plugin can be~interpreted as~a standard plugin. Since the~transformation is~done by~executing a~SPARQL query, it~corresponds with an~execution of~a SPARQL query plugin. During the~evaluation of~the analytical pipeline we~reach a~certain point of~an undergoing mapping (thanks to~a DCV plugin). On~the output of~this plugin, we~will for certain find a~dataset compliant with the~DCV standard. But the~dataset need not 
be the~result of~the analytical pipeline execution. It~may be~used as~an input 
for the~upcoming analytical steps.

Based on~facts presented in~the text above we~find it~reasonable to~integrate 
the proposed system into~Payola.

\section{Formalization}

In this section, we~are going to~formalize the~aforementioned process a~bit 
more. We~need to~specify the~input of~the system. We~will also 
provide a~definition of~the mapping process. A~description of~the output will be~also presented.

\subsection{Input and output}

The input of~the system will be~an arbitrary RDF graph $G_{input}$ as~shown
in Figure~\ref{fig:mapping-example}. It~will be~mapped into~a~form compliant
with the~DCV standard. Since the~DCV also takes advantage of~the RDF standard,
the result will likewise be~a graph, let us~say $G_{output}$.

\begin{figure}
	\centering
	\includegraphics[width=120mm]{img/definition-in-graph.png}
	\caption{A user will provide an~arbitrary RDF graph $G_{def}$, which may contain more than one definition. The~user will select only one to~work with, $D$.}
	\label{fig:definition-in-graph}
\end{figure}

As stated before, the~system will have another input --- a~data structure 
definition. The~Data Cube Vocabulary standards are built on~top of~the RDF as~described in~the 
section~\ref{datacube-vocabulary}. That means, that a~data structure is~also a~graph. We~will denote the~user--provided definition as~$D$. But the~user will probably not 
supply just a~definition. They will provide an~arbitrary graph, which contains the~definition $D$ 
(or even more definitions) as~a subgraph (see Figure~\ref{fig:definition-in-graph}).
We will denote a~graph
supplied by~the user as~$G_{def}$. The~system has to~extract a~list of~all the~definitions
(let us~say $D_1, ..., D_d$) from the~graph $G_{def}$. As~shown in~Figure~\ref{fig:mockup-02},
the user will select one of~the offered definitions, for instance $D = D_x (1 \leq x~\leq d)$.
The graph may contain also other data, not only DCV definitions. It~is true 
that:\\

{\centering $D \subseteq G_{def}$ \\[0.5cm]}

But having the~graphs $G_{input}$ and $D$ 
is not enough. We~require the~user to~specify also an~example pattern (based on~the
query--by--example principle). Since all the~inputs and outputs are RDF
graphs ($G_{input}$, $G_{def}$, $D$, $G_{output}$), we~can take 
advantage of~some existing approaches specialized on~RDF transformation. 
A technically simple but powerful way, is~to utilize the~SPARQL language. Based on~what we~have introduced before, we~are able to~transform a~user--selected pattern into~a~form of~a SPARQL
query. We~will talk about such a~query as~of a~\emph{transformation query} and denote it~$Q_t$.

What we~have now is~an information about what is~needed on~the input and 
what the~result would be:

{\centering $(G_{input}, G_{def}, Q_t) \rightarrow G_{output}$ \\[0.5cm]}

In fact, the~user will make the~selection of~the definition $D$ before the~mapping starts. Therefore, we~can extract the~definition $D$ from the~generic graph $G_{def}$ and use it~as a~part of~the input. That is~possible because 
the system does not require an~arbitrary RDF graph $G_{def}$. It~requires the~DCV 
data structure definition $D$. The~rest is~done in~order to~improve the~user 
experience. Therefore:\\

{\centering $(G_{input}, G_{def}, Q_t) \rightarrow (G_{input}, D, Q_t) \rightarrow G_{output}$  \\[0.5cm]}

What we~still do~not have is~a description of~the SPARQL query used to~transform the~data in~order to~obtain the~output graph $G_{output}$. The~form of~the query is~dependent on~the 
definition $D$. To~be more accurate, the~definition $D$ contains
a set of~components ($C_1, ... , C_n$), which represents dimensions, measures and attributes. Let us~say, that the~structure has defined $n$ components, $m$ measures, $d$ dimensions and $a$ 
attributes. It~is true that\\

{\centering $n = m+d+a$ \\[0.5cm]}

For each n--tuple matched in~$G_{input}$, the~output graph $G_{output}$ will contain a~node with $n+2$ 
neighbours (+1 for \verb|qb:Observation| and +1 for \verb|qb:dataSet| --- first two statements
in Figure~\ref{fig:output-graph}). A~concrete example of~an observation is~shown 
in Figure~\ref{fig:output-graph-instance}.

The result will comply with the~template shown
in Figure~\ref{fig:output-graph}. $O_i$ denotes an~identifier (URI) of~the generated 
observation, $S$ stands for an~identifier of~the input dataset,
$T_{Dim_1}$~...~$T_{Dim_d}$ stands for the~type of~a corresponding dimensions 
from the~given data structure definition, similarly for
$T_{Msr_1}$~...~$T_{Msr_m}$ -- measures and
$T_{Attr_1}$~...~$T_{Attr_a}$ -- attributes. $R_{i,1}$~...~$R_{i,n}$ stands for
resources matched in~the input graph $G_{input}$, moreover, it~means that the~value 
of the~dimension $T_{Dim_1}$ is~$R_{i,1}$, etc.

\begin{figure}
  \centering
  \begin{tabular}{lll}
$O_i$~~~~~~~~~~~~& a~~~~~~~~~~~~~~~~~~~~~~~~~& qb:Observation ;\\
          & qb:dataSet    & $S$ ;\\
          & $T_{Dim_1}$ & $R_{i,1}$ ; \\
          & $...$              & $...$ \\
          & $T_{Dim_d}$  & $R_{i,k}$ ; \\
          & $T_{Msr_1}$  & $R_{i,l}$ ; \\
          & $...$              & $...$ \\
          & $T_{Msr_m}$ & $R_{i,m}$ ; \\
          & $T_{Attr_1}$  & $R_{i,o}$ ; \\
          & $...$              & $...$ \\
          & $T_{Attr_a}$  & $R_{i,n}$ . \\
\end{tabular}
\caption{Description of~an output graph node (turtle notation)}
\label{fig:output-graph}
\end{figure}

\begin{figure}
  \centering
  \scriptsize
  \begin{tabular}{lll}
\textless http://ex.com/observed\#1234\textgreater & a& qb:Observation~;\\
          & qb:dataSet    &  \textless http://dbpedia.org/sparql\textgreater ~;\\
          & my-population:count & '10233222' ; \\
          & my-population:location & \textless http://dbpedia.org/page/Prague\textgreater ~;\\
          & my-population:time  & '2012-02-23' . \\
  \end{tabular}
\caption{A concrete instance of~the pattern shown in~Figure~\ref{fig:output-graph}}
\label{fig:output-graph-instance}
\end{figure}

\subsection{Transformation query}

The last unanswered question is~how the~query $Q_t$ should look like. 
It would certainly not be~an arbitrary query. Let us~analyze the~situation and 
walk the~reader through the~process of~coming up~with a~set of~constraints for the~query.

It should be~obvious that the~query $Q_t$ will generate a~new graph
(in fact, it~will generate the~graph $G_{output}$) based on~some specific rules
(see Figure~\ref{fig:mapping-example}). 
Therefore, it~would be~a CONSTRUCT query. It~is also very clear what kind of~triples it~will generate; the~pattern is~shown in~Figure~\ref{fig:output-graph}.
The only thing we~need to~sort out is~retrieving the~resources $R_{i,1}, ... R_{i,n}$.

\subsubsection{Pattern}
\label{sec:pattern-definition}
The process of~resolving those resources depends on~the input graph $G_{input}$.
We will demonstrate it~on the~most simple case --- a~2--dimensional data structure 
definition. Such a~definition requires us~to specify how to~select a~3--tuple of~resources from 
the graph (2 dimensions, 1 measure).
We will present some ideas while using the~example of~the population size.

The initial idea could be~that we~need a~country, therefore, we~map all the~countries to~the \texttt{refArea} dimension, all instances of~time to~the \texttt{refPeriod} 
dimension and all instances of~\emph{numbers} to~the measure. But that is~not sufficient. At~first,
we need to~make sure that all instances of~numbers 
really express an~amount of~citizens in~a country. Moreover, we~need to~map the
matching amount of~citizen to~a corresponding country. Therefore, all the~selected resources need to~be connected with the~rest. We~need to~select a~specific pattern, which will contain the~relation between the~resources.

\begin{figure}
	\centering
	\includegraphics[width=80mm]{images/cherry.png}
	\caption{Cherry--shaped pattern example.}
	\label{fig:cherry}
\end{figure}

\begin{figure}
	\centering
	\includegraphics[width=80mm]{images/cherry_blank.png}
	\caption{The example from Figure~\ref{fig:cherry} extended with a~blank node.}
	\label{fig:cherry-blank}
\end{figure}

We will use an~example from DBPedia, 
the Czech Republic resource. The~resource itself reference to~a location, 
therefore, it~should be~mapped to~the \verb|refArea| dimension. It~has got a~\verb|populationTotal| property, which should be~mapped to~the measure. Last, but not 
least, it~has the~\verb|populationAsOf| property, which should be~mapped to~the 
\verb|refPeriod| dimension.

That gives us~a simple shape of~a cherry--pair as~shown in
Figure~\ref{fig:cherry}. But what 
if the~resource us~connected to~a blank node with an~edge of~type 
population and the~blank node has two properties --- size and time
(see Figure~\ref{fig:cherry-blank})?
It is~clear now that there could be~in the~least a~long oriented path between the~participating 
vertices.

While keeping this fact in~mind, we~can accede to~assembling a~formula of~a generic pattern used for an~extraction of~the resources from the~original graph $G_{input}$. But 
there is~one piece of~a puzzle still missing. While analyzing all the
possibilities, until now we~assumed that the~pattern \emph{starts} (in the~topological order)
with a~resource, which represents one of~the data structure components. But that is~not  
the case.

The most simple example is~the observation pattern itself. The~entities come together but their connection is~dependent on~a completely 
different resource. We~reference such an~entity
as a~\emph{reference vertex}. Any pattern vertices corresponding to~a component from the~data structure definition could also match the~\emph{reference vertex}.

Later on, we~have discovered that such a~kind of~patterns is~not what the~user 
wants to~select. We~have applied too much restrictions. Such a~pattern
represents only a~subset of~patterns the~user may want to~select. We~have learnt
that the~most concrete designation of~such a~pattern is~the well--known term:
\emph{connected graph}.

We left out the~term \emph{oriented} on~purpose. While experimenting with the~system, we~had discovered that for purposes of~the pattern specification the~orientation is~not necessary. We~need all the~referenced vertices to~have a~relation between each other, but the~orientation does not matter.

In order to~get the~reader familiar with some 
approaches we~will utilize while implementing the~system, we~are about to~point 
out some facts.

We are about to~build a~pattern $P$. The~pattern reflects an~example set by~the user,
an example that represents the~user’s proceedings of~the data mapping to~the components from a~data structure definition. The~pattern will
be used later to~construct the~desired
transformation query $Q_t$. As~shown in~Figure~\ref{fig:mapping-example}, we~require the~user to~find an~equivalent for each component ($\{C_1, ... C_n\}$) of~the data
structure definition $D$ in~the input graph $G_{input}$. The~equivalent for $C_1$ is~in 
Figure~\ref{fig:output-graph} denoted as~$R_{i,1}$. But there are many other 
equivalents for the~component in~the input graph ($R_{i,1}$ is~one of~many instances). 
That is~why we~need to~generalize the~symbols used for matching 
resources. Therefore, we~introduce a~new set of~symbols, $M_1, ..., M_n$ to~express that such a~resource is~a vertex matched in~the mapping process, $M_1$ 
for component $C_1$, etc. To~understand this better, see Figure~\ref{fig:mapping-pattern}}.

\begin{figure}
	\centering
	\includegraphics[width=140mm]{img/mapping-pattern.png}
	\caption{An example of~mapping. $G_{input}$ is~an input graph, $D$ a~data 
	structure definition, $G_{output}$ an~output graph. Lines between those graphs represent mapping.
	For each component $C_1, ..., C_n$ we~need the~user to~give an~example of~a matching entity.
	That means one concrete instance $R_{i,1}$ for component $C_1$. Such match is~generally denoted
	as $M_1$.}
	\label{fig:mapping-pattern}
\end{figure}

Please, keep in~mind that we~are about to~construct a~connected graph. Therefore, we~define an~invariant, which says that in~every 
step of~the pattern construction the~pattern $P$ needs to~be a~connected graph.

As shown in~Figure~\ref{fig:mockup-03} and Figure~\ref{fig:mockup-05},
the system iterates over data 
structure definition components and lets the~user select an~example of~a 
vertex corresponding to~the current component.

For each component we~are about 
to extend the~constructed pattern. We~will take advantage of~the notation from the
graph theory, because the~pattern is~also a~graph (let us~remember the~connected graph invariant).
Therefore, in~the beginning:\\

{\centering $P = (V = \emptyset, E~= \emptyset)$ \\[0.5cm]}

We will extend the~pattern $P$ for each component $C_1, ..., C_n$ of~the 
definition $D$. We~need the~pattern to~contain all the~vertices from $\{M_1 ,..., 
M_n\}$, because they represent examples of~each component. An~example of
a pattern extension for a~component is~shown in~Figure~\ref{fig:pattern-enhancement}.

In order to~maintain the~invariant, we~will enable the~user to~add only such 
vertices that are somehow related with an~arbitrary vertex already contained in~the pattern $P$ (an element of~set of~vertices $V$ --- vertices $a,b,c,d$
in Figure~\ref{fig:pattern-enhancement}, step 1).
To make it~more formal, let us~remind the~reader that an~edge is~a tuple 
of vertices (e.g. $(v_1,v_2)$). In~case of~an oriented graph, it~depends on~the order within 
the tuple.

Therefore when adding an~arbitrary vertex $v_i$, we~demand that there exists an~edge $e$ for which it~is true that:

\begin{center}
{$e = (v_i,v_x)$ or~$e = (v_x, v_i)$ \land $v_x \in V$ \\[0.5cm]}
\end{center}

\begin{figure}
	\centering
	\includegraphics[width=140mm]{img/pattern-enhancement.png}
	\caption{An example of~pattern extension. We~start with three selected vertices
	-- $v_1, v_2, v_3$. Due to~the invariant, we~are able to~extend the~pattern only with
	the vertices $a,b,c,d$. We~choose the~vertex $d$.}
	\label{fig:pattern-enhancement}
\end{figure}

Of course, this is~not possible to~apply in~the first step since the~graph $P$ is~empty. 
In order to~make the~pattern more accurate or~more restrictive to~obtain a~more 
efficient transformation query, the~user may want to~reference some other 
vertices and not only the~component equivalents.
Because of~that observation, for each component from $C_i \in \{C_1 ,..., C_n\}$ the~pattern is~extended by~adding a~set of~vertices $V_i = \{v_{i_1}, ..., v_{i_u}\}$ and set of~corresponding edges 
$E_i$:\\

{\centering \forall $C_i \in \{C_1 ,..., C_n\}$: $P = (V = V~\cup V_i, E~= E~\cup E_i)$ \\[0.5cm]}

In Figure~\ref{fig:pattern-enhancement} we~make two steps. In~the first one, we~can add vertices $a,b,c,d$ and we~choose to~add $d$. That means that we~extend 
the set of~vertices $V$ with the~set $V_i = \{v_4 = d\}$ and the~set of~edges $E$ with 
the set $E_i = \{(v_1,v_4 = d)\}$.

Because of~the invariant:\\

{\centering $\exists e~\in E_i: e~= (v_{in}, v_{out}) \land (v_{in} \in V~\lor v_{out} \in V)$\\[0.5cm]}

and of~course:\\

{\centering $M_i \in V_i \subseteq V$ \\[0.5cm]}

--- the~example equivalent $M_i$ for component $C_i$ was also selected. One of~the 
added vertices from $V_i = \{v_{i_1}, ..., v_{i_u}\}$ is~equal to~$M_i$ ($v_3$ is~equal to~$M_1$
in Figure~\ref{fig:pattern-enhancement}).
   
By applying the~described approach, we~get a~connected graph, where:\\

{\centering $\{M_1, ..., M_n\} \subset V$ .\\[0.5cm]}

That means, that the~pattern covers our needs. Now, we~are about to~present
a generic form of~the pattern, which will come out from 
such an~approach. The~pattern is~shown in~Figure~\ref{fig:sparql-pattern}. 
$n$ of~the vertices from the~$\{v_1, ..., v_v\}$ matches vertices $M_1, ..., M_n$.

\begin{figure}
\begin{Verbatim}[commandchars=\\\{\},codes={\catcode`$=3\catcode`_=8}]
  CONSTRUCT \{
    []   a~  <http://purl.org/linked-data/cube#Observation> ;
         <http://purl.org/linked-data/cube#dataSet>   $S$ ;
         $C_1$      $M_1$ ;
         ...   
         $C_i$      $M_i$ ;
         ...   
         $C_n$      $M_n$ .
  \} WHERE \{
    \{
      SELECT DISTINCT ?$M_1$ ... ?$M_n$
      \{
         $v_1$     $E_1$     $v_2$
         $v_2$     $E_2$     $v_3$
         ...
         $v_{i-1}$   $E_{i-1}$    $v_i$
         $v_i$     $E_i$      $v_{i+1}$
         ...
         $v_{v-1}$   $E_{v-1}$    $v_v$
      \}
    \}
  \}
\end{Verbatim}
\caption{Pattern of~the transformation SPARQL query}
\label{fig:sparql-pattern}
\end{figure}

\section{Visualizers}
In the~beginning, we~promised to~implement an~exemplary visualization.
Unfortunately, the~Data Cube Vocabulary standard covers a~wide variety of~data domains, 
therefore it~is hard to~make a~decision and choose, which visualization should be~offered.

Based on~what we~have seen while exploring tools described in~Chapter~\ref{chap:rw} and keeping in~mind the~rules of~a so--called 
\emph{visualisation mantra} (Section~\ref{sec:rw:mantra}}), we~decided to~implement two visualizers, 
TimeHeatmap and Universal DCV. 

\subsection{TimeHeatmap}
The decision to~implement this visualizer was based on~the example of~the population size data
presented in~Chapter~\ref{ch:statistical-data}. A~table or~a bar chart is~definitely a~reliable way of~presenting this kind of~data. But while getting 
familiar with Data Cube Vocabulary, we~found out that it~is very popular to~visualize geospatial data. A~visualization on~a map is~easy to~understand and is
very popular among the~non--technical people. It~helps to~popularize Data Cube 
Vocabulary. This is~also the~correct type of~visualization for data journalists we~have mentioned before.

As the~name suggests, this visualizer is~able to~handle datasets with two 
kinds of~dimensions. The~first one will express the~time of~the measurement, the~second one will cover its location. It~will support one measure that has to~be 
a number.

We will place a~heatmap layer over a~standard map layer. The~layer will express 
the intensity of~the measured value in~respect to~the others. The~scale will go~from 
green to~red where the~latter represents the~largest value measured.

\subsection{Universal DCV}
Implementing a~domain--complete library of~visualizers is~a long--term project. 
Despite the~fact, we~would like to~offer a~visualizer, which would enable the~user to~visualize a~large amount of~datasets in~a comfortable way. While
experimenting we~have experienced on~our own some discomfort in~reading triple tables.

That is~why we~have decided to~implement a~visualizer, which takes advantage of~the idea behind faceted browsers. We~presented some of~those in~Chapter~\ref{chap:rw}. Therefore we~would like to~implement a~visualizer that will 
enable the~user to~slice visualized datasets and prepare usual visualizations of~the slices. A~mockup of~such a~visualizer is~shown in~Figure~\ref{fig:dcv-universal}. It~should contain a~pie chart and a~bar chart 
visualization.

\begin{figure}
	\centering
	\includegraphics[width=140mm]{img/dcv-universal.png}
	\caption{Data Cube Vocabulary universal visualizer.}
	\label{fig:dcv-universal}
\end{figure}



