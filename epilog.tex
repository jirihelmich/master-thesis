\chapter*{Conclusion}
\addcontentsline{toc}{chapter}{Conclusion}

We have implemented the~proposed system. We~have thusly introduced a~tool which is, with
a user input, able to~convert an~arbitrary RDF dataset into~a~form compliant with Data Cube Vocabulary.
As a~part of~our long--term research, we~decided to~integrate it~into~a~larger system (Payola)
in order to~benefit from such an~integration.
For example, the~user is~able to~make the~conversion as~a~part of~a~larger 
analytical process. It~enables the~user to~work with the~statistical data in~a~usual way. Moreover, they are able to~work with data from multiple datasets at~once.

We have also examined several already existing tools but did not find one enabling
the user to~perform such a~conversion. There were some related tools 
focused on~a~very similar task and we~used to~our advantage the~knowledge gained from
examining them. Especially, we~used the~query--by--example principle in~order to~deliver our system.

During the~process of~implementation, we~have managed to~extend Payola, our RDF application. 
We not only added the~Data Cube Vocabulary related features but also included some 
other useful user experience improvements. Moreso, we~integrated a~completely 
new approach (in the~scope of~Payola) a~construction of~a~SPARQL query, which can 
be reused in~other scenarios in~order to~deliver some new features.

Based on~our experience and our research, we~have also determined those parts of~Payola, which are in~need of~improvement or~redesign for providing a~more 
efficient behaviour. That is~also crucial in~order to~improve the~reliability, 
the performance and the~usability of~the~implemented system while working on~this 
thesis.

Last, but not least, we~delivered two new visualizer plugins, which enable the~user to~explore the~Data Cube Vocabulary related datasets. Whilst the~first one tends to~be~universal, the~second one is~focused on~visualizing geospatial data. Improving 
the user experience of~those visualizers should become the~core of~the~future work, 
including an~implementation of~a~fully faceted browser. To~offer a~visualizer,
which may probably be~considered a~basis for a~sophisticated faceted browser for cube data,
represents a~significant achievement.

We fulfilled the~goal of~this thesis. In~spite of~the~fact that the~system will be~eventually improved
as it~happens with every other new system, we~find
the implementation satisfactory to~the~current state of~the~Payola framework. We~consider the~provided feature to~be~of~a~great benefit to~the~Payola users.
